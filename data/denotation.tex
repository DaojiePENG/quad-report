% !TeX root = ../sustechthesis-example.tex

% denotation 环境带一个可选参数,用来指定符号列的宽度(默认为 2.5cm),下面改3cm为例。
% 如果论文中使用了大量的物理量符号、标志、缩略词、专门计量单位、自定义名词和术语等, 应编写“符号和缩略语说明”。
% 论文中主要符号应全部采用法定单位, 严格执行《量和单位》(GB3100~3102-93)的有关规定、单位名称的书写,可以采用国际通用符号,也可以用中文名称,但全文应统一,不得两种混用。
% 缩略语应列出中英文全称。符号和缩略语说明排序方法先按拉丁字母大写、小写排序, 再按希腊字母大写、小写排序, 如下表所示:
% ABCDEFGHIJKLMNOPQRSRUVWXYZ
% abcdefghijklmnopqrstuvwxyz
% Alpha
% Beta
% Gamma
% Delta
% Epsilon
% Zeta
% Eta
% Theta
% Iota
% Kappa
% Lambda
% Mu
% Nu
% Xi
% Omicron
% Pi
% Rho
% Sigma
% Tau
% Upsilon
% Phi
% Chi
% Psi
% Omega
% alpha
% beta
% gamma
% delta
% epsilon
% zeta
% eta
% theta
% iota
% kappa
% lambda
% mu
% nu
% xi
% omicron
% pi
% rho
% sigma
% tau
% upsilon
% phi
% chi
% psi
% omega

% 希腊字母详见 https://xilazimu.net/


\begin{denotation}[3cm]
  \item[As-PPT]聚苯基不对称三嗪
  \item[DFT]密度泛函理论 (Density Functional Theory)
  \item[DMAsPPT]聚苯基不对称三嗪双模型化合物(水解实验模型化合物)
  \item[$E_a$]化学反应的活化能 (Activation Energy)
  \item[HMAsPPT]聚苯基不对称三嗪模型化合物的质子化产物
  \item[HMPBI]聚苯并咪唑模型化合物的质子化产物
  \item[HMPI]聚酰亚胺模型化合物的质子化产物
  \item[HMPPQ]聚苯基喹噁啉模型化合物的质子化产物
  \item[HMPY]聚吡咙模型化合物的质子化产物
  \item[HMSPPT]聚苯基对称三嗪模型化合物的质子化产物
  \item[HPCE]高效毛细管电泳色谱 (High Performance Capillary lectrophoresis)
  \item[HPLC]高效液相色谱 (High Performance Liquid Chromatography)
  \item[IRC]内禀反应坐标 (Intrinsic Reaction Coordinates)
  \item[LC-MS]液相色谱-质谱联用 (Liquid chromatography-Mass Spectrum)
  \item[MAsPPT]聚苯基不对称三嗪单模型化合物,3,5,6-三苯基-1,2,4-三嗪
  \item[MPBI]聚苯并咪唑模型化合物,N-苯基苯并咪唑
  \item[MPI]聚酰亚胺模型化合物,N-苯基邻苯酰亚胺
  \item[MPPQ]聚苯基喹噁啉模型化合物,3,4-二苯基苯并二嗪
  \item[MPY]聚吡咙模型化合物
  \item[MSPPT]聚苯基对称三嗪模型化合物,2,4,6-三苯基-1,3,5-三嗪
  \item[ONIOM]分层算法 (Our own N-layered Integrated molecular Orbital and molecular Mechanics)
  \item[PBI]聚苯并咪唑
  \item[PDT]热分解温度
  \item[PES]势能面 (Potential Energy Surface)
  \item[PI]聚酰亚胺
  \item[PMDA-BDA]均苯四酸二酐与联苯四胺合成的聚吡咙薄膜
  \item[PPQ]聚苯基喹噁啉
  \item[PY]聚吡咙
  \item[S-PPT]聚苯基对称三嗪
  \item[SCF]自洽场 (Self-Consistent Field)
  \item[SCRF]自洽反应场 (Self-Consistent Reaction Field)
  \item[TIC]总离子浓度 (Total Ion Content)
  \item[TS]过渡态 (Transition State)
  \item[TST]过渡态理论 (Transition State Theory)
  \item[ZPE]零点振动能 (Zero Vibration Energy)
  \item[\textit[ab initio]]基于第一原理的量子化学计算方法,常称从头算法
  \item[$\Delta G^\neq$]活化自由能(Activation Free Energy)
  \item[$\kappa$]传输系数 (Transmission Coefficient)
  \item[$\nu_i$]虚频 (Imaginary Frequency)
\end{denotation}



% 也可以使用 nomencl 宏包,需要在导言区
% \usepackage{nomencl}
% \makenomenclature

% 在这里输出符号说明
% \printnomenclature[3cm]

% 在正文中的任意为都可以标题
% \nomenclature{As-PPT}{聚苯基不对称三嗪}
% \nomenclature{DFT}{密度泛函理论 (Density Functional Theory)}
% \nomenclature{DMAsPPT}{聚苯基不对称三嗪双模型化合物(水解实验模型化合物)}
% \nomenclature{$E_a$}{化学反应的活化能 (Activation Energy)}
% \nomenclature{HMAsPPT}{聚苯基不对称三嗪模型化合物的质子化产物}
% \nomenclature{HMPBI}{聚苯并咪唑模型化合物的质子化产物}
% \nomenclature{HMPI}{聚酰亚胺模型化合物的质子化产物}
% \nomenclature{HMPPQ}{聚苯基喹噁啉模型化合物的质子化产物}
% \nomenclature{HMPY}{聚吡咙模型化合物的质子化产物}
% \nomenclature{HMSPPT}{聚苯基对称三嗪模型化合物的质子化产物}
% \nomenclature{HPCE}{高效毛细管电泳色谱 (High Performance Capillary lectrophoresis)}
% \nomenclature{HPLC}{高效液相色谱 (High Performance Liquid Chromatography)}
% \nomenclature{IRC}{内禀反应坐标 (Intrinsic Reaction Coordinates)}
% \nomenclature{LC-MS}{液相色谱-质谱联用 (Liquid chromatography-Mass Spectrum)}
% \nomenclature{MAsPPT}{聚苯基不对称三嗪单模型化合物,3,5,6-三苯基-1,2,4-三嗪}
% \nomenclature{MPBI}{聚苯并咪唑模型化合物,N-苯基苯并咪唑}
% \nomenclature{MPI}{聚酰亚胺模型化合物,N-苯基邻苯酰亚胺}
% \nomenclature{MPPQ}{聚苯基喹噁啉模型化合物,3,4-二苯基苯并二嗪}
% \nomenclature{MPY}{聚吡咙模型化合物}
% \nomenclature{MSPPT}{聚苯基对称三嗪模型化合物,2,4,6-三苯基-1,3,5-三嗪}
% \nomenclature{ONIOM}{分层算法 (Our own N-layered Integrated molecular Orbital and molecular Mechanics)}
% \nomenclature{PBI}{聚苯并咪唑}
% \nomenclature{PDT}{热分解温度}
% \nomenclature{PES}{势能面 (Potential Energy Surface)}
% \nomenclature{PI}{聚酰亚胺}
% \nomenclature{PMDA-BDA}{均苯四酸二酐与联苯四胺合成的聚吡咙薄膜}
% \nomenclature{PPQ}{聚苯基喹噁啉}
% \nomenclature{PY}{聚吡咙}
% \nomenclature{S-PPT}{聚苯基对称三嗪}
% \nomenclature{SCF}{自洽场 (Self-Consistent Field)}
% \nomenclature{SCRF}{自洽反应场 (Self-Consistent Reaction Field)}
% \nomenclature{TIC}{总离子浓度 (Total Ion Content)}
% \nomenclature{TS}{过渡态 (Transition State)}
% \nomenclature{TST}{过渡态理论 (Transition State Theory)}
% \nomenclature{ZPE}{零点振动能 (Zero Vibration Energy)}
% \nomenclature{\textit{ab initio}}{基于第一原理的量子化学计算方法,常称从头算法}
% \nomenclature{$\Delta G^\neq$}{活化自由能(Activation Free Energy)}
% \nomenclature{$\kappa$}{传输系数 (Transmission Coefficient)}
% \nomenclature{$\nu_i$}{虚频 (Imaginary Frequency)}
