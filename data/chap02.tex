% !TeX root = ../sustechthesis-example.tex

\chapter{基于强化学习的机械狗控制}

\section[单层神经网络直接驱动关节方式]{\label{section:direct_network}单层神经网络直接驱动关节方式\cite[p3]{Hwangbo_Lee_Dosovitskiy_Bellicoso_Tsounis_Koltun_Hutter_2019}}

\textcolor{red}{\small
基于参考文献阐述\emph{单层神经网络直接驱动关节方式}控制方法...
}

\section[双层神经网络本体感知方式]{双层神经网络本体感知方式\cite[p8]{Hwangbo_Lee_Dosovitskiy_Bellicoso_Tsounis_Koltun_Hutter_2019}}

\textcolor{red}{\small
基于参考文献阐述\emph{双层神经网络本体感知方式}控制方法...
}

\section[双层网络本体和外部感知融合方式]{双层网络本体和外部感知融合方式\cite[p7]{Lee_Hwangbo_Wellhausen_Koltun_Hutter_2020}}

\textcolor{red}{\small
基于参考文献阐述\emph{双层网络本体和外部感知融合方式}控制方法...
}

\section[电机驱动关节点模型]{电机驱动关节点模型\cite[p4]{Gehring_Coros_Hutter_Bellicoso_Heijnen_Diethelm_Bloesch_Fankhauser_Hwangbo_Hoepflinger_et_al_2016}}

\textcolor{red}{\small
基于参考文献阐述\emph{电机驱动关节点模型}描述...
}
