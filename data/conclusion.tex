% !TeX root = ../sustechthesis-example.tex

\begin{conclusion}

对于经典控制的实现,涉及到复杂的运动模型和动力学建模,难点在于选择优化目标构建优化策略和优化问题的求解。经典控制的优点在于控制过程清晰,可以方便地进行迭代优化和泛化部署;缺点在于建模和实际机器调试复杂,运行时算力开销大。对于强化学习控制的实现,涉及到机器人的仿真模型建模和神经网络设计,难点在于设计合适的优化目标和训练策略以使得训练过程能较快收敛且结果达到预期目标。强化学习控制的优点在于模型设计和调试应用简单迅速,运行时算力开销小;缺点在于控制过程不清晰,导致模型的泛化能力较弱。

经典控制与深度学习控制各有优缺点,在实践应用中两者的有机结合有助于实现更好的控制。在面对一种机器人设计控制策略时,如何分配经典控制和深度学习控制是一个十分重要的问题。两者的融合也是未来的重要发展方向之一。


\end{conclusion}
