% !TeX root = ../sustechthesis-example.tex

\begin{conclusion}

对于经典控制的实现,涉及到复杂的运动模型和动力学建模,难点在于选择优化目标构建优化策略和优化问题的求解。经典控制的优点在于控制过程清晰,可以方便地进行迭代优化和泛化部署;缺点在于建模和实际机器调试复杂,运行时算力开销大。对于强化学习控制的实现,涉及到机器人的仿真模型建模和神经网络设计,难点在于设计合适的优化目标和训练策略以使得训练过程能较快收敛且结果达到预期目标。强化学习控制的优点在于模型设计和调试应用简单迅速,运行时算力开销小;缺点在于控制过程不清晰,导致模型的泛化能力较弱。

经典控制与深度学习控制各有优缺点,在实践应用中两者的有机结合有助于实现更好的控制。在面对一种机器人设计控制策略时,如何分配经典控制和深度学习控制是一个十分重要的问题。两者的融合也是未来的重要发展方向之一。

对于接下来的研究,就当前的情况来看,实现强化学习到实际机器人\emph{ANYmal}上的部署还欠缺的是:
\begin{enumerate}
    \item 一套ANYmal机器人硬件;
    \item 可用的例程代码;
    \item 电机、控制手柄、各类传感器的驱动和软件接口。
    \begin{itemize}
        \item 在有直接可用例程的情况下,可以采用例程的架构和电机接口直接部署如\ref{section:direct_network}节所述的神经网络策略。
        \item 在没有直接可用例程有类似参考例程的情况下,可以花一些时间阅读相关代码,逐步将电机、控制手柄、各类传感器等驱动重构到ANYmal上,然后在此基础上设计控制结构实现策略部署。
        \item 在没有任何软件接口的情况下至少应该有各种器件相应的手册。会消耗比较长的时间来完成驱动和信息格式的整理和构建。完成后设计控制结构进行策略部署。
    \end{itemize}
    \item 对Isaac仿真环境的深入学习;
    \item 对PyTorch框架的深入学习;
\end{enumerate}

对于后续的研究,在上面的基础上需要进一步学习和尝试将经典策略应用到机器人上,以更加深入的理解经典控制方法。与此同时加强对强化学习理论的理解和仿真环境的使用能力,在接下来的研究中构思\emph{经典-强化学习融合}的控制策略。在此基础上,考虑设计和实现新结构的机器人。
\end{conclusion}
