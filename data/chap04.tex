% !TeX root = ../sustechthesis-example.tex

\chapter{几种构想的机器人类型草图}
这里给出几种近来构想的机器人结构草图和简单介绍,可以作为随后研究的机器人备选平台参考。
\section[弹跳轮足+机械手机器人]{弹跳轮足+机械手机器人}
该类型的机器人由类似于Ascento\cite[p1]{Klemm_Morra_Salzmann_Tschopp_Bodie_Gulich_Kung_Mannhart_Pfister_Vierneisel_et_al_2019}类型机器人的双轮足,在此基础上添加两个机械臂构成类人形机器人。该类型的机器人可以轮式地行走和跳跃,同时可以用机械臂模仿双手进行各种操作。

\textcolor{red}{\small
待补充示意图片...
}

该类型机器人的控制基础是倒立摆模型和机械臂运动学,它相对于现有双轮机器人,如Ascento\cite[p1]{Klemm_Morra_Salzmann_Tschopp_Bodie_Gulich_Kung_Mannhart_Pfister_Vierneisel_et_al_2019}等,新的控制问题难点是:
\begin{enumerate}
  \item 在机械手运动的同时保持双轮足的稳定,比如从空手到搬起重物的过程;
  \item 如何让手臂配合轮足实现更加灵巧和优雅的动作;
\end{enumerate}

\section[方块轮足+机械手机器人]{方块轮足+机械手机器人}
该类型的机器人由类似一般人形机器人的基本结构,在此基础之上为每个脚添加四个轮子

\textcolor{red}{\small
待补充示意图片...
}

该类型机器人的控制基础是ZMP约束和机械臂运动学,它相对于现有人形机器人,如Xxx等,新的控制问题难点是:
\begin{enumerate}
  \item 如何实现步态和轮式之间的运动切换;
  \item 如何实现步态和轮式的联合运动平衡;
  \item 在上面两点的基础上,如何结合机械臂实现更加灵巧、优雅、节能的动作;
\end{enumerate}

\section[弹跳轮足+飞行器机器人]{弹跳轮足+飞行器机器人}
该类型的机器人由类似于Ascento\cite[p1]{Klemm_Morra_Salzmann_Tschopp_Bodie_Gulich_Kung_Mannhart_Pfister_Vierneisel_et_al_2019}类型机器人的双轮足,在此基础之上添加一对旋翼使其具备飞行能力。

\textcolor{red}{\small
待补充示意图片...
}

该类型机器人的控制基础是倒立摆模型和飞行器控制,它相对于现有双轮机器人,如Ascento\cite[p1]{Klemm_Morra_Salzmann_Tschopp_Bodie_Gulich_Kung_Mannhart_Pfister_Vierneisel_et_al_2019}等,新的控制问题难点是:
\begin{enumerate}
  \item 飞行器的矢量控制;
  \item 如何优化设计使得旋翼提供的升力可以满足飞行控制
  \item 续航安全问题;
  \item 如何让旋翼配合轮足实现更加灵巧、优雅、节能、稳定的动作;
\end{enumerate}

\section[机械狗+飞行器机器人]{机械狗+飞行器机器人}
该类型的机器人由类似于ANYmal\cite[p1]{Hwangbo_Lee_Dosovitskiy_Bellicoso_Tsounis_Koltun_Hutter_2019}类型机器人的四足机器人,在此基础之上添加两对旋翼使其具备飞行能力。

\textcolor{red}{\small
待补充示意图片...
}

该类型机器人的控制基础是ZMP约束和飞行器控制,它相对于现有双轮机器人,如Ascento\cite[p1]{Klemm_Morra_Salzmann_Tschopp_Bodie_Gulich_Kung_Mannhart_Pfister_Vierneisel_et_al_2019}等,新的控制问题难点是:
\begin{enumerate}
  \item 飞行器的矢量控制;
  \item 如何优化设计使得旋翼提供的升力可以满足飞行控制
  \item 续航安全问题;
  \item 如何让旋翼配合轮足实现更加灵巧、优雅、节能、稳定的动作;
\end{enumerate}