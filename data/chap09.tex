% !TeX root = ../sustechthesis-example.tex

\chapter[讨论]{讨论}
整体的感觉是这个领域的研究在没有人指导的情况下一个人突围是比较费劲的,会比较耗时。就当前的情况来看,经典学习的问题求解和控制到机器人上的部署是比较困难的;而RL虽然仿真起来看似容易,但它的训练策略设置和策略的\emph{仿真到实际(sim-to-real)}部署也比较困难。如果能有找到人指导一下或有机会到成熟的组里学习一段时间,实际上手项目来先\textbf{建立一个整体的研究流程感受}是比较好的。

对于接下来的研究,实现强化学习到实际机器人\emph{ANYmal}上的部署还欠缺的有:
\begin{enumerate}
    \item 一套ANYmal机器人硬件;
    \item 可用的例程代码;
    \item 电机、控制手柄、各类传感器的驱动和软件接口。
    \begin{itemize}
        \item 在有直接可用例程的情况下,可以采用例程的架构和电机接口直接部署如\ref{section:direct_network}节所述的神经网络策略。
        \item 在没有直接可用例程有类似参考例程的情况下,可以花一些时间阅读相关代码,逐步将电机、控制手柄、各类传感器等驱动重构到ANYmal上,然后在此基础上设计控制结构实现策略部署。
        \item 在没有任何软件接口的情况下至少应该有各种器件相应的手册。会消耗比较长的时间来完成驱动和信息格式的整理和构建。完成后设计控制结构进行策略部署。
    \end{itemize}
    \item 对Isaac仿真环境的深入学习,一方面提高对它的理解深度和使用能力,另一方面也需跟进了解其他仿真平台的发展;
    \item 对PyTorch框架的深入学习,加深对算法的掌握,积累策略设计的经验;
\end{enumerate}

对于后续的研究,在上面的基础上需要进一步学习和尝试将经典策略应用到机器人上,以更加深入的理解经典控制方法。与此同时加强对强化学习理论的理解和仿真环境的使用能力,在接下来的研究中构思\emph{经典-强化学习融合}的控制策略。在此基础上,考虑设计和实现新结构的机器人。