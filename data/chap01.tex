% !TeX root = ../sustechthesis-example.tex

\chapter{基于经典方式的机械狗控制}

足式机器人的运动规划是比较困难的,不仅因为它的自由度较多,更因为它的机体运动不能被直接得出,而是要通过四肢状态及四肢与环境的接触产生。

经典控制的基础是对机器人的运动学和动力学建模。经典控制主要方式有\emph{轨迹优化(Trajectory Optimization, TO)}和\emph{模型预测控制(Model Predictive Control, MPC)}两类。常用的\emph{轨迹优化}机械狗动力模型有:\emph{1. 基于有线性倒立摆的模型;2. 基于直接刚体动力学的模型;3. 基于质心动力学的模型}。基于这些模型的优化控制都被描述为\emph{非线性优化问题},这些问题的求解算力开销较大往往不能够在线完成,需要上位机的辅助。
\textcolor{red}{\small
对于MPC还不太熟悉,等看完了补充下MPC描述...
}


\section[基于线性倒立摆模型的控制]{基于线性倒立摆模型的TO控制\cite[p2-5]{Bellicoso_Jenelten_Fankhauser_Gehring_Hwangbo_Hutter_2017}}
\textcolor{red}{\small
基于参考文献阐述基于ZMP模型的控制方法...
}


\subsection[机械狗的运动模型]{机械狗的运动模型}
\textcolor{red}{\small
基于参考文献阐述机械狗的一般运动学表述...
}
\section[基于质心动力学模型的控制]{基于质心动力学模型的TO控制\cite[p2-6]{Winkler_Bellicoso_Hutter_Buchli_2018}}

\textcolor{red}{\small
基于参考文献阐述基于接触力模型的控制方法...
}

\section[基于直接刚体动力学模型的控制]{基于直接刚体动力学模型的TO控制\cite[p2-6]{Pardo_Neunert_Winkler_Grandia_Buchli_2017}}
\textcolor{red}{\small
基于参考文献阐述基于刚体动力学模型的控制方法...
}
\section[基于MPC的控制]{基于MPC的控制\cite[p2-4]{Neunert_Stauble_Giftthaler_Bellicoso_Carius_Gehring_Hutter_Buchli_2018}}

\textcolor{red}{\small
基于参考文献阐述基于MPC的控制方法...
}